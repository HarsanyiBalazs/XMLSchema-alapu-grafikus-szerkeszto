\Chapter{Koncepció}

\\Section{Mi az XML}

Az XML (Extensible Markup Language)egy egyszerű és rugalmas jelölőnyelv és fájlformátum, amit adatok rendezésére, tárolására és szállítására használnak. A XML Working Group (eredetileg SGML Editorial Review Board) 1996-ban kezdte el fejlesztését, aminek első verziója 1998-ben jelent meg, a W3C ajánlásával. A tervezés során arra törekedtek, hogy az XML-feldolgozó programok könnyen elkészíthetők legyenek, és az XML számos különféle alkalmazással kompatibilisen működjön. A dokumentumoknak gyorsan és egyszerűen kialakíthatóknak kell lenniük, továbbá fontos szempont volt az is, hogy az ember számára olvashatók maradjanak.

Egy XML dokumentum nem előre, hanem egyénileg definiált címkékből áll, ezeknek lehetnek értékkel ellátott attribútumai és tartalma. Egy címke tartalma, állhat szövegből,további címkékből vagy lehet üres is. A címke és attribútumok tetszés szerint nevezhetőek és a címkék bármilyen komplexitási szintre beágyazhatóak. Az XML kizárólag az információ rendszerezésére összpontosít.

\Section{XML sémák}

Az XML sémák olyan opcionális leírások vagy fájlok, amellyel szabályozni lehet egy XML dokumentum  szerkezetét és tartalmát. A séma tartalmazhatja, hogy milyen elemek és attribútumok szerepelhetnek egy XML-ben, milyen típusú adatok megengedettek, milyen sorrendben jelenhetnek meg, illetve melyek kötelezők vagy opcionálisak. Egy XML dokumentum csak akkor tekinthető érvényesnek, ha a séma összes elvárásának megfelel.

Az XML sémák azért lényegesek, mert garantálják az XML-fájlok egységes felépítését és könnyű feldolgozhatóságát. A séma meghatározza a dokumentum szerkezetét, így biztosítva, hogy a program helyesen tudja értelmezni, és minimalizálva a futás közbeni hibák kockázatát. Emellett a validálás segítségével gyorsan és egyszerűen kiszűrhetők a dokumentumban előforduló hibák.

Az első sémanyelv, a DTD (document type definition) már valamilyen formában az XML megjelenése előtt is jelen volt, mivel az SGML nyelvcsalád általános használatához volt kifejlesztve, így csak a meglévő nyelvet kellet átdolgozni. A DTD definiálható saját fájlban, de akár az XML dokumentumban is. A DTD alkalmas megadni az elemeket, azoknak a tartalmát, számosságát és sorrendjét, az attríbútomokat, és ezeken felül entitásokkal hivatkozásokat lehet készíteni. Bár az alapvetők megtalálhatóak, a nyelvnek nagy hiányosságai vannak. Kevés az adattípusok száma és nem képes újabb, komplexebb típusokat definiálni, így nem lehet megkötéseket tenni a szövegértékeknek. A névterek kezelésére sem képes, így a több forrásból származó XML állományok kezelésében problémás.További hátránya, hogy a szintaxisa eltér az XML-től, így más XML-t kezelő eszközökkel nem használható. Mindezek miatt viszonylag hamar megjelentek az újabb, fejlettebb sémanyelvek, amelyek fokozatosan kiszorították a DTD használatát.


%https://www.xml.com/pub/a/w3j/s3.bosak.html

%https://www.w3.org/TR/REC-xml/

%https://www.w3schools.com/xml/xml_dtd_intro.asp
